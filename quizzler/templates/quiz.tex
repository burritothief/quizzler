\documentclass{article}
\usepackage{xcolor}
\usepackage{calc}
\usepackage[width = 18cm]{geometry}
\usepackage{enumitem}
\usepackage[nodayofweek]{datetime}
\usepackage{color,fancyhdr,ifthen,amssymb,amsfonts,amsmath}
\usepackage[tikz]{multicolrule}
\usepackage{hyperref}
\usepackage{caption}
\pagestyle{fancy}

\hypersetup{
    hidelinks,
    pdftitle={SAT Reading \& Writing Quiz},
    pdfauthor={Jeff},  
    pdfsubject={SAT study and test preparation},
    bookmarks=true,
    bookmarksopen=true,   
    bookmarksnumbered=true,      
    % bookmarksopenlevel=1,                 
    % pdfstartview={Fit},           
    % pdfpagemode={UseOutlines},  
}

\setlength{\headheight}{12.09006pt}
\setlength{\topmargin}{-.5in} \setlength{\textheight}{9.25in}
% \setlength{\columnsep}{15pt}
% \setlength{\oddsidemargin}{0in} \setlength{\textwidth}{6.8in}

\newcounter{question}
\newcounter{choice}
\setcounter{choice}{1}
\newenvironment{question}[1][]{%
  \stepcounter{question}%
  \noindent\colorbox{black}{\textcolor{white}{\makebox[1.5em][c]{
    \hyperlink{#1-A}{\hypertarget{{#1}}{\thequestion}}
  }}}%
  \colorbox{lightgray}{\phantom{\rule{\columnwidth-1.5em-\columnsep}{\heightof{1}}}\makebox[0pt][r]{\textit{\footnotesize#1}}}\par
  \begin{list}{}{%
    \setlength\itemindent{0pt}%
    \setlength\leftmargin{2em}%
    \setlength\rightmargin{0.5em}%
    \setlength\labelwidth{0pt}%
    \setlength\labelsep{0pt}}%
    }%
{\end{list}}

\newenvironment{choices}{%
  \begin{list}{\Alph{choice})}{%
    \setlength\labelwidth{30pt}%
    \setlength\itemindent{0pt}%
    \setlength\leftmargin{2em}%
    \setlength\labelsep{10pt}%
    \usecounter{choice}}}%
  {\end{list}}
\setlength\columnseprule{0.5pt}

% For adjusting the margins around excerpt block quotes 
% https://latex.org/forum/viewtopic.php?p=30994&sid=4526e0a87d9595b4599f9e0ed162271c#p30994
\renewenvironment{quote}{%
  \list{}{%
    \leftmargin0.5cm
    \rightmargin\leftmargin
  }
  \item\relax
}
{\endlist}

\lhead{\bf{MMSF}}
\rhead{\shortdate\today}
\cfoot{\bf{\thepage}}


\begin{document}
\pdfbookmark{Quiz}{quiz}
\begin{multicols}{2}
  \raggedcolumns
  \SetMCRule{line-style=dots}
  <BLOCK> for q in questions </BLOCK>
  <BLOCK> if q.prompt.__class__.__name__ == 'BasicPrompt' </BLOCK>
  \begin{question}[<VAR>q.identifier</VAR>]
    \begin{samepage}
      \item[] <VAR> q.prompt.text | set_uniform_underscore_length | tex_escape </VAR>
      \item[] <VAR> q.prompt.question | tex_escape </VAR>
      \begin{choices}
        <BLOCK> for choice in q.choices </BLOCK>
        \item <VAR> choice | tex_escape </VAR>
        <BLOCK> endfor </BLOCK>
      \end{choices}
    \end{samepage}
  \end{question}
  <BLOCK> elif q.prompt.__class__.__name__ == 'ExercptPrompt' </BLOCK>
  \begin{question}[<VAR>q.identifier</VAR>]
    \begin{samepage}
      \item[] <VAR>q.prompt.pretext | tex_escape </VAR>
      \begin{quote}
        <VAR>q.prompt.text | set_uniform_underscore_length | tex_escape </VAR>
      \end{quote}
      \item[] <VAR>q.prompt.question | tex_escape </VAR>
      \begin{choices}
        <BLOCK> for choice in q.choices </BLOCK>
        \item <VAR> choice | tex_escape </VAR>
        <BLOCK> endfor </BLOCK>
      \end{choices}
    \end{samepage}
  \end{question}
  <BLOCK> elif q.prompt.__class__.__name__ == 'CrossTextPrompt' </BLOCK>
  \begin{question}[<VAR>q.identifier</VAR>]
    \begin{samepage}
      \item[] \textbf{Text 1}
      \item[] <VAR>q.prompt.text1 | tex_escape </VAR>
      \item[] \textbf{Text 2}
      \item[] <VAR> q.prompt.text2 | tex_escape </VAR>
      \item[] <VAR> q.prompt.question | tex_escape </VAR>
      \begin{choices}
        <BLOCK> for choice in q.choices </BLOCK>
        \item <VAR> choice | tex_escape </VAR>
        <BLOCK> endfor </BLOCK>
      \end{choices}
    \end{samepage}
  \end{question}
  <BLOCK> elif q.prompt.__class__.__name__ == 'RhetoricalAnalysisPrompt' </BLOCK>
  \begin{question}[<VAR>q.identifier</VAR>]
    \begin{samepage}
      \item[] <VAR>q.prompt.pretext | tex_escape </VAR>
      \renewcommand\labelitemi{{\boldmath$\cdot$}}
      \begin{itemize}
        \setlength\labelsep{5pt}%
        <BLOCK> for note in q.prompt.notes </BLOCK>
        <BLOCK> if note is iterable and note is not string </BLOCK>
        \begin{itemize}
          <BLOCK> for subnote in note </BLOCK>
          \item <VAR> subnote | tex_escape </VAR>
                <BLOCK> endfor </BLOCK>
        \end{itemize}
        <BLOCK> else </BLOCK>
        \item <VAR> note | tex_escape </VAR>
              <BLOCK> endif </BLOCK>
              <BLOCK> endfor </BLOCK>
      \end{itemize}
      \item[] <VAR> q.prompt.question | tex_escape </VAR>
      \begin{choices}
        <BLOCK> for choice in q.choices </BLOCK>
        \item <VAR> choice | tex_escape </VAR>
        <BLOCK> endfor </BLOCK>
      \end{choices}
    \end{samepage}
  \end{question}
  <BLOCK> else </BLOCK>
  <BLOCK> endif </BLOCK>
  <BLOCK> endfor </BLOCK>
\end{multicols}

\vspace*{\fill}
\begin{center}
  \huge\textbf{STOP}
\end{center}

%%%%%%%%%%%%%%
% Answer Key %
%%%%%%%%%%%%%%
\newpage
\onecolumn
\pdfbookmark{Answer Key}{answer-key}
\section*{Answer Key}
\begin{enumerate}
  <BLOCK> for q in questions </BLOCK>
  \item \hypertarget{<VAR>q.identifier</VAR>-A}{\hyperlink{<VAR>q.identifier</VAR>}{\textbf{(<VAR>q.solution.choice.to_string()</VAR>)}}} <VAR> q.solution.explanation | tex_escape </VAR>
        <BLOCK> endfor </BLOCK>
\end{enumerate}
\end{document}




